\documentclass[landscape,12pt]{report}
\usepackage{anysize}
\usepackage{graphicx}
\usepackage{hyperref}
\marginsize{1.5cm}{1.5cm}{1.0cm}{1.0cm}
\usepackage{graphicx}
\usepackage{subfig}
\begin{document}

    \begin{tabular}{c c c}
            %------------------------------------%
            %-------------Columna No. 1----------%
            %------------------------------------%
            \begin{minipage}[t]{8.2cm} 
                        \begin{center}
                \textbf{\Large Partes Principales de una Computadora}
                \end{center}
                    Una computadora se compone principalmente del software y el hardware. Algunos de los componentes b\'asicos m\'as importantes de una laptop son:
\begin{itemize} 
\item Disco duro
\item Memoria RAM
\item Procesador
\item Motherboard (Tarjeta madre)
\end{itemize}
\begin{center}
\includegraphics[width=7cm]{./foto1.jpg}
\end{center}










                   
            \end{minipage}& 
            %------------------------------------%
            %-------------Columna No. 2----------%
            %------------------------------------%
            \begin{minipage}[t]{8.2cm}%comienza la segunda pagina

                \begin{center}
                \textbf{\Large Computadora}
                \end{center}
               Las computadoras en la actualidad son  herramientas de gran utilidad para realizar diversas tareas de acuerdo a las necesidades de las personas. Siendo las computadoras herramientas con tantas capacidades, necesitan de cuidados y mantenimientos para su \'optimo funcionamiento.
                \newline\newline
                \textbf{�Qu\'e es una Computadora?}
                \newline\newline
                La computadora es un dispositivo electr\'onico que fue dise�ado con el prop\'osito de procesar la datos que el usuario ingresa y mediante diferentes m\'etodos devolverla convertida en informaci\'on \'util que el operador necesita. Para lograrlo se utilizan programas, es decir el software o aplicaciones. 
                                  \newline\newline
                Las computadoras se han vuelto una herramienta de gran importancia tanto a nivel educativo como nivel laboral. El valor de una computadora radica en la velocidad y precisi\'on con la cual \'esta ejecuta las operaciones o instruciones. La capacidad de una computadora se hace espec\'ificamente para el ambiente en donde \'esta va a operar.
                \end{minipage} & 
    
            %------------------------------------%
            %-------------Columna No. 3----------%
            %------------------------------------%
                \begin{minipage}[t]{8.2cm} %comienza la tercera pagina
                    
                    \vspace{2cm}
                    %Portada del trifoliar 
                    \begin{center}
                       \title{\textbf{Mantenimiento de una Computadora} \\ \large Practicas Iniciales e Intermedias Segundo Semestre}
                         \author{Robin Salvatierra , 200915428 \\
Herbert Reyes , 201612114 \\ 
Javier Lima, 201612098
}

                         \date{\today}
                        \maketitle
                    \end{center}                
                
                \end{minipage}
\end{tabular}
                                
                                
          %-------------------------------------------------------------%
          %-------------------Segunda P\'agina del trifoliar--------------%
          %-------------------------------------------------------------%
\newpage 

\begin{tabular}{||c ||c|| c||}
            %------------------------------------%
            %-------------Columna No. 1----------%
            %------------------------------------%
            \begin{minipage}[t]{8.2cm}
                
                \begin{center}
                \title{\textbf{\large Mantenimiento de la Computadora}} 
                \end{center}
                
             Una laptop/notebook acumula polvo en su exterior pero tambi\'en acumula suciedad dentro de ella, lo c\'ual no solo afecta el rendimiento de la m\'aquina si no puede ocasionar un deterioro mayor al sobrecalentarse los chips integrados como el de v\'ideo o el procesador. Una vez ocasionado el deterioro es necesario una sustituci\'on del dispositivo.
                \begin{center}
                \title{\textbf{\large Materiales Necesarios Para el Mantenimiento}} 
                \end{center}
                \begin{itemize}
                       \item Desarmadores
                       \item Espuma Limpiadora
                       \item Aire Comprimido
                       \item Limpia Contactos
                       \item Brocha
                       \item Toalla
                       \item Pasta T\'ermica
                \end{itemize}
                                  \begin{center}
                    \includegraphics[width=4cm]{./limpieza.jpg}
                    \end{center}

            \end{minipage}& 
            %------------------------------------%
            %-------------Columna No. 2----------%
            %------------------------------------%
            \begin{minipage}[t]{8.2cm}                
  
              \begin{center}
                \title{\textbf{\large Medidas de Seguridad}} 
                \end{center}
                    Estas medidas son necesarias, debido a que son vitales para la seguridadtanto del equipo como la del t�cnico.
                    \begin{itemize}
                    \item Antes de abrir cualquier computadora, es necesaria ver si tiene alguna falla.
                    \item Antes de quitar los tornillos remover toda fuente de alimentaci\'on
                    \item Utilizar el desarmador adecuado.
                    \item Trabajar en un lugar limpio.
                    \item Nunca forzar ningun componente.
                    \end{itemize}

  
    \begin{center}
                \title{\textbf{\large Desensamblamiento de la Computadora}} 
                \end{center}

\begin{enumerate}
\item Remover el teclado.
\item Quitar los tornillos debajo del teclado.
\item Retirar todos los tornillos de la parte inferior.
\item Destornilar y retirar la placa de la tarjeta RAM y el disco duro.
\item Retirar la memoria RAM y la unidad \'optica.
\item Retirar el disco duro.
\item Quitar la parte inferior de la computadora.
\end{enumerate}
                                
                \end{minipage} & 
            %------------------------------------%
            %-------------Columna No. 3----------%
            %------------------------------------%
                \begin{minipage}[t]{8.2cm}
 \begin{center}
                \title{\textbf{\large Limpieza de los componentes}} 
                \end{center}

\begin{enumerate}
\item Retirar el ventilador del procesador.
\item Limpiar y sustituir la pasta t\'ermica antigua.
\item Limpiar todo el polvo que pueda estar acumulado en la motherboard.. 
\item Utilizar el  limpia contactos en toda la motherboard, y esperar hasta que se seque. 
\item Limpiar el polvo en el ventilador.
\item Colocar la parte inferior de la computadora.
\item Atornillar la parte inferior, colocar la memoria RAM, el disco duro y la unidad \'optica y colocar la placa.
\item Colocar los tornillos debajo del teclado y luego colocar el teclado.
\item Limpiar el teclado con aire comprimido y brocha.
\item Aplicar espuma y limpiar las partes externas de la computadora.
\item Encender y probar la computadora.
\end{enumerate}
            
              \end{minipage}
            
\end{tabular}

\end{document}
